\documentclass[11pt,a4paper]{beamer}
\usepackage[ngerman]{babel}
\usepackage[T1]{fontenc}
\usepackage[utf8]{inputenc}
\usepackage{amsmath}
\usepackage{amsfonts}
\usepackage{amssymb}

% Notwendig für Gantt-Diagramme (Zeitstrahl)
\usepackage{tikz}
\usepackage{gantt}

\usetheme{Berlin}
\author{Julian Baumann, Xenia Kühling, Sebastian Ruder}
\title{Koreferenz mit BART}
\subtitle{Spezifikationsvortrag zum Softwareprojekt im Sommersemester 2014}
\date{27. Mai 2014}

\begin{document}
\maketitle

\section{Einführung}
\begin{frame}{Inhalt}
\tableofcontents
\end{frame}

\begin{frame}{evtl Koreferenz}
\begin{itemize}
\item Erklärung Koreferenz
\end{itemize}
\end{frame}

\section{BART}
\begin{frame}{BART}
\begin{itemize}
\item Beautiful Anaphora Resolution Toolkit
\item Entstanden im Projekt\\ \textit{Exploiting Lexical and Encyclopedic Resources For Entity Disambiguation} 
im John Hopkins Summer Workshop 2007
\item System für automatische Koreferenzresolution
\item Weiterentwicklungen im Rahmen von shared tasks, für verschiedene Sprachen
\end{itemize}
\end{frame}

\begin{frame}{Wie funktioniert BART}
\begin{itemize}
\item Modularer Aufbau: 
\item Vorverarbeitungsphase
\item Extraktion NP- Kandidaten, NP- Merkmale, Kandidatenpaare
\end{itemize}
\end{frame}

\begin{frame}{Wie funktioniert BART}
\begin{itemize}
\item Resolution mit Soon Algorithmus 
\item Kandidatenpaare werden paarweise anhand ihrer Merkmale verglichen
\item Ergebnisse
\end{itemize}
\end{frame}

\begin{frame}{Aufgabe}
\begin{itemize}
\item Resolution in BART mit System der Stanford-NLP-Gruppe
\item Bestes Ergebnis bei CoNLL-2011 shared task, obwohl hauptsächlich regelbasiert
\end{itemize}
\end{frame}

\section{Stanford Sieves}
\begin{frame}
efregrgoiu
\end{frame}

\section{Aufgaben und Module}
\section{Zeitplan}
    \begin{gantt}{10}{9}
    \begin{ganttitle}
      \titleelement{Mai}{1}
      \titleelement{Juni}{4}
      \titleelement{Juli}{4}
    \end{ganttitle}
    \begin{ganttitle}
      \titleelement{27.05.}{1}
      \titleelement{03.06.}{1}
      \titleelement{10.06.}{1}
      \titleelement{17.06.}{1}
      \titleelement{24.06.}{1}      
      \titleelement{01.07.}{1}
      \titleelement{08.07.}{1}
      \titleelement{15.07.}{1}
      \titleelement{22.07.}{1}
    \end{ganttitle}
    \ganttbar{Pipeline läuft}{0}{2}
    \ganttmilestone[color=cyan]{1. \textit{Sieve} läuft}{2}
    \ganttbar{\textit{Sieves} einfügen}{2}{3}
    \ganttmilestone[color=cyan]{\textit{Sieves} laufen}{5}
    \ganttbar{Evaluation}{5}{2}
    \ganttbar[color=cyan]{Bugfixes}{2}{5}
    \ganttbar{Präsentation}{7}{1}
    \ganttbar[color=cyan]{Dokumentation}{5}{4}
  \end{gantt}
  



\section{Softwarespezifikation}
\section{Quellen}


\end{document}