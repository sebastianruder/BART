\documentclass[11pt,a4paper]{beamer}
\usepackage[latin1]{inputenc}
\usepackage[german]{babel}
\usepackage{amsmath}
\usepackage{amsfonts}
\usepackage{amssymb}
\usetheme{Berlin}
\author{Julian Baumann, Xenia K\"uhling, Sebastian Ruder}
\title{Koreferenz mit BART}
\subtitle{Spezifikationsvortrag zum Softwareprojekt im Sommersemester 2014}
\date{27. Mai 2014}

\begin{document}
\maketitle

\section{Einf\"uhrung}
\begin{frame}{Inhalt}
\tableofcontents
\end{frame}

\begin{frame}{evtl Koreferenz}
\begin{itemize}
\item Erkl\"arung Koreferenz
\end{itemize}
\end{frame}

\section{BART}
\begin{frame}{BART}
\begin{itemize}
\item Beautiful Anaphora Resolution Toolkit
\item Entstanden im Projekt\\ \textit{Exploiting Lexical and Encyclopedic Resources For Entity Disambiguation} 
im John Hopkins Summer Workshop 2007
\item System f\"ur automatische Koreferenzresolution
\item Weiterentwicklungen im Rahmen von shared tasks, f\"ur verschiedene Sprachen
\end{itemize}
\end{frame}

\begin{frame}{Wie funktioniert BART}
\begin{itemize}
\item Modularer Aufbau: 
\item Vorverarbeitungsphase
\item Extraktion NP- Kandidaten, NP- Merkmale, Kandidatenpaare
\end{itemize}
\end{frame}

\begin{frame}{Wie funktioniert BART}
\begin{itemize}
\item Resolution mit Soon Algorithmus 
\item Kandidatenpaare werden paarweise anhand ihrer Merkmale verglichen
\item Ergebnisse
\end{itemize}
\end{frame}

\begin{frame}{Aufgabe}
\begin{itemize}
\item Resolution in BART mit System der Stanford-NLP-Gruppe
\item Bestes Ergebnis bei CoNLL-2011 shared task, obwohl haupts\"achlich regelbasiert
\end{itemize}
\end{frame}

\section{Stanford Sieves}
\begin{frame}
efregrgoiu
\end{frame}

\section{Aufgaben und Module}
\section{Zeitplan}
\section{Softwarespezifikation}
\section{Quellen}


\end{document}