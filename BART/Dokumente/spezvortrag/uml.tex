\documentclass[12pt]{article}

\usepackage{url}
\usepackage{epsfig}
\usepackage{graphicx}
\usepackage{lingmacros}
\usepackage{tree-dvips}
%\usepackage{avm}
%\bibpunct{(}{)}{,}{a}{,}{,}
\addtolength{\textwidth}{3cm}
\addtolength{\textheight}{5cm}
\addtolength{\oddsidemargin}{-2cm}
\addtolength{\topmargin}{-1cm}

\parindent=0cm

\usepackage[german]{babel}
\usepackage[T1]{fontenc}
\usepackage[latin1]{inputenc} 

\usepackage{tikz-dependency}
\usepackage{qtree}
\usepackage{pgf}

\newcommand{\bit}{\begin{itemize}}
\newcommand{\eit}{\end{itemize}}
\newcommand{\bt}{\begin{tabular}}
\newcommand{\et}{\end{tabular}}
\newcommand{\mc}{\multicolumn}
\newcommand{\absatz}{\vspace*{3mm}}
\newcommand{\ul}{\underline}
\newcommand{\dn}{\downarrow}
\newcommand{\up}{\uparrow}


\begin{document}
\noindent
{\sc
\begin{tabular}{@{}l@{\hspace*{94mm}}r@{}}
\hline\\[-2mm]
{\bf Formale Syntax}&3.5.2012 \\
Prof. Dr. Anette Frank&\\
SoSe 2012 & \\[1mm]
\hline
\end{tabular}
}

\subsection*{Aufgabenblatt (Abgabe bis 13.5.2012)}

\subsubsection*{Aufgabe 1 \hfill (5 Punkte)}

Zeichnen sie zwei m\"ogliche F-Strukturen f\"ur den folgenden Satz.
Beschreiben Sie verschiedene Situationen, auf die Sie mit den entsprechenden
syntaktischen Lesarten Bezug nehmen k\"onnen.

\enumsentence{{\em Peter parkte das Auto auf der Wiese neben der Autobahn.}}

\subsubsection*{Aufgabe 2}

Erkunden Sie mit Hilfe der online-LFG Grammatiken im XLE Web Interface
\url{http://iness.uib.no/iness/xle-web}, wie computationelle
LFG-Grammatiken folgende zentralen Ph\"anomene der Syntax analysieren.

{\small
\begin{enumerate}

\item Passiv: {\em Die Gesellschaft wurde durch einen Schauer aufgescheucht.}
\item Verbstellung:

 {\em Fritz vermutet, dass Maria einen Liebhaber hat.}\\
			   {\em Peter kam um 5 Uhr in Heidelberg an.}
\item Reflexiv: 

{\em Fritz rasiert sich nicht.}\\
{\em Maria \"angstigt sich.}
\item Komplement/Adjunkt: 

{\em Fritz liest den ganzen Spiegel.\\
 Fritz liest den ganzen Tag. \\
\# Der ganze Tag wurde gelesen. \\
Den ganzen Tag wurde gelesen.}
\item Raising/Kontrolle

{\em Fritz droht die Beziehung zu beenden.\\
Maria verspricht Hans sich zu bessern.}
\item Satzeinbettung

{\em Fritz l\"ugt, ohne rot zu werden. \\Wenn Maria ein Fahrrad h\"atte, w\"are sie oft in der Stadt.}
\item Lange Abh\"angigkeiten

{\em Wen hat Fritz geglaubt, im Kino gesehen zu haben?}
\item Koordination

{\em Fritz und Maria gehen gern ins Kino. \\
Peter kommt nach Hause und stellt das Radio an.}
\end{enumerate}
}
\subsubsection*{Aufgabe 3 \hfill (15 Punkte)}


Erstellen Sie auf Papier eine kleine Grammatik und Lexikoneintr\"age,
die es erlauben, die folgenden S\"atze zu analysieren:

\eenumsentence{ 
\item {\em  Gestern traf Maria einen Hochstapler.}
\item {\em Maria gab ihm Geld.}}
				
Zeichnen Sie die C- und F-Strukturen f\"ur beide S\"atze und geben Sie die notwendigen annotierten C-Strukturregeln an.                 
Wenden Sie den Resolutionsalgorithmus auf Satz (1.b) an, um Ihre
vorgeschlagene F-Struktur formal zu verifizieren.

\paragraph{Hinweise:}

\begin{itemize}
\item Adverbien wie {\em gestern} werden als Element einer Adjunktmenge
repr\"asentiert. 
\item Funktionale Annotationen f\"ur die Einbettung einer
F-Struktur als Element einer Menge werden wie folgt kodiert:

$\downarrow \in$ ($\uparrow$ {\sc ADJ}): die
F-Struktur, auf die $\uparrow$ abgebildet wird, enth\"alt ein Attribut {\sc ADJ},
dessen Wert eine Menge ist. Element dieser Menge ist eine F-Struktur, auf
die $\downarrow$ abgebildet wird.

\item F\"ur Personalpronomina verwenden Sie die C-Struktur-Kategorie PRON. 

\item F\"ur Artikel verwenden Sie die Kategorie DET.

 In der F-Struktur verwenden Sie f\"ur die Art des Artikels das
 Merkmal {\sc SPEC}, als Werte: def, indef, poss, etc.
\item 
F\"ur die Verbalphrase verwenden Sie eine flache Struktur (mit fakultativem Verb):

\Tree  [.VP  XP \dots (V) ]

\end{itemize}

\end{document}